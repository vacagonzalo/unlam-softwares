\documentclass[
    11pt,
    spanish,
    a4paper
]{article}
\usepackage[utf8]{inputenc}
\usepackage[spanish]{babel}
\usepackage{authoraftertitle}
\usepackage{booktabs}
\usepackage{caption}
\usepackage{float}
\usepackage{graphicx}
\usepackage{listings}
\usepackage{verbatim}

\def\doctype{INFORME DE INVESTIGACIÓN}
\title{Softwares de simulación para Tecnología electrónica y otras cátedras de ingeniería}
\author{Gonzalo Nahuel Vaca}

\begin{document}

\makeatletter
\begin{titlepage}
	\begin{center}
		\vspace*{1cm}

		\Huge
		\textbf{\doctype}
		\vspace{0.5cm}

		\LARGE
		\@title
		\vspace{0.5cm}

		\textbf{Compatibilidad electromagnética}

		\vspace{1.5cm}

		\textbf{\@author}

		\vspace{1.5cm}

		\includegraphics[width=0.4\textwidth]{img/logoUNLaM.jpeg}

		\vfill
		Maestría en Sistemas Embebidos\\
		Universidad de Buenos Aires\\
		Argentina\\
		\today
	\end{center}
\end{titlepage}
\makeatother
\newpage

\section{Resumen}

Este informe es una memoria breve del trabajo realizado durante el año 2022.

\section{OpenEMS}

El programa OpenEMS fue escrito en el lenguaje C++ y utiliza el método de las diferencias finitas en el dominio temporal (FDTD) para resolver simulaciones electromagnéticas.

El programa fue pensado originalmente para realizar simulaciones de equipos médicos de resonancia magnética (MRI).
Sin embargo, se desarrolló un analizador léxico que permite transformar el formato \emph{Hyperlynx} que utiliza KiCad al esquema de geometría de OpenEMS.

El modelo geométrico de OpenEMS es \emph{CSXCAD} y es una biblioteca propia del programa.
Además, permite su visualización y trabajo en coordenadas cartesianas y cilíndricas.
Finalmente, provee una interfaz para Matlab/Octave y Python.

Las características más relevantes son:

\begin{itemize}
	\item Diseño y simulación en coordenadas cartesianas.
	\item Diseño y simulación en coordenadas cilíndricas.
	\item Soporta modelos \emph{voxel} para simulaciones MRI.
	\item Multi-hilos y SIMD.
	\item Intefaz Matlab/Octave.
	\item Definición de materiales en el espacio.
	\item Definición de excitaciones en el espacio.
	\item Generación de archivos en formato \emph{vtk} y \emph{hdf5}
	\item Materiales dispersivos(Drude/Lorentz/Debye)
	\item Rutinas de post-procesamiento configurables (Matlab/Octave).
	\item Sub-espacios de simulación para optimizar densidades de cálculo.
	\item Simulaciones remotas por medio de \emph{ssh}.
	\item Es libre y gratuito.
\end{itemize}

Se creó un entorno de trabajo en Ubuntu 2020 y en contenedores de \emph{docker} para las versiones de Ubuntu 2018 y 2016.
Además se estudió como utilizarlo con Octave y se obtuvieron los siguientes resultados:

\begin{itemize}
	\item El analizador léxico se encuentra abandonado y no es posible hacerlo funcionar
	\item KiCad 6 cambió su formato \emph{Hyperlynx} y el analizador léxico es por lo tanto obsoleto.
	\item Las geometrías de los circuitos impresos se deben construir a mano.
	\item Los errores de geometría no son útiles para depurar los datos ingresados.
\end{itemize}

Dado el estado actual de este programa se hace imposible utilizarlo como herramienta de las cátedras si antes no se repara y actualiza el analizador léxico.

\section{Ansys}

Ansys HFSS (simulador de estructura de alta frecuencia) es un solucionador comercial de métodos de elementos finitos para estructuras electromagnéticas (EM) de Ansys que ofrece varias tecnologías de resolución de última generación.
Cada solucionador en ANSYS HFSS es un procesador de solución automatizado para el cual el usuario dicta la geometría, las propiedades del material y el rango requerido de frecuencias de solución.

Los ingenieros utilizan Ansys HFSS principalmente para diseñar y simular componentes electrónicos de alta velocidad y alta frecuencia en sistemas de radar, sistemas de comunicación, satélites, ADAS, microchips, placas de circuito impreso, productos IoT y otros dispositivos digitales y dispositivos RF.
El solucionador también se ha utilizado para simular el comportamiento electromagnético de objetos como automóviles y aviones.
ANSYS HFSS permite a los diseñadores de sistemas y circuitos simular problemas de EM, como pérdidas por atenuación, acoplamiento, radiación y reflexión.

HFSS captura y simula objetos en 3D, teniendo en cuenta la composición de los materiales y las formas/geometrías de cada objeto.
Además, está preparada para ser utilizadas en el diseño de antenas y elementos complejos de circuitos electrónicos de radiofrecuencia, incluidos filtros, líneas de transmisión y empaques.

Durante el transcurso del 2022 se logró un contacto comercial que realizó una propuesta de trabajo a la UNLaM a cambio de licencias y capacitación.
Se dió una capacitación sobre la funcionalidad y sinergia entre los distintos motores de físicas.

\section{Elmer}

Elmer es un software de simulación multi-física de código abierto desarrollado principalmente por \emph{IT Center for Science} (CSC).
El desarrollo de Elmer se inició como una colaboración nacional con las universidades finlandesas, los institutos de investigación y la industria.

Elmer incluye modelos físicos de dinámica de fluidos, mecánica estructural, electromagnetismo, transferencia de calor y acústica.
Estos se describen mediante ecuaciones diferenciales parciales que Elmer resuelve mediante el método de elementos finitos (FEM).
Elmer admite computación paralela.

Actualmente los campos de uso más destacados son la glaciología computacional y el electromagnetismo computacional.

\section{Conclusiones}

\begin{itemize}
	\item OpenEMS: no se recomienda para su uso ni continuar su investigación.
	\item Ansys: se recomienda continuar con su investigación.
	\item Elmer: se debe comenzar su investigación.
\end{itemize}

\end{document}